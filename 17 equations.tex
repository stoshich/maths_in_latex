\documentclass{article}
\usepackage[utf8]{inputenc}
\usepackage[russian]{babel}
\title{17 уравнений, которые изменили мир}
\author{Иан Стюарт }
\date{}
\begin{document}
\maketitle
\begin{tabbing}
1.\ \ \ \=Теорема пифагора \ \ \ \=$a^2+b^2=c^2$ \ \ \ \ \ \ \ \ \ \ \ \ \ \ \ \ \ \ \ \ \ \ \ \ \ \ \ \ \ \ \ \ \ \ \=Пифагор, 530 год до н. э.\\
2.\>Логарифмы \>$\log xy = \log x + \log y$ \>Джон Напьер, 1610 год\\
3.\>Бесконечно малые \>$\frac{df}{dt} = \lim \limits_{h\to 0} \frac{f(t+h)-f(t)}{h}$ \>Ньютон, 1668 год\\
4.\>Закон всемирного \>$F = G\frac{m_{1} m_{2}}{r^2}$\>Ньютон, 1687 год\\
\>тяготения\\
5.\>Мнимая единица\>$i^{2} = -1$\>Эйлер, 1750 год\\
6.\>Формула Эйлера\>$\mathrm{V}-\mathrm{E}+\mathrm{F}=2$\>Эйлер, 1751 год\\
\>для многогранника\\
7.\>Нормальное\>$\phi \left(x\right)=\frac{1}{\sqrt{2\pi p}}{C}^{\frac{{\left(x-p\right)}^{2}}{2{p}^{2}}}$\>К. Ф. Гаусс, 1810 год\\
\>распределение\\
8.\>Волновое уравнение\>$\frac{{\partial }^{2}u}{\partial {t}^{2}}={C}^{2}\frac{{\partial }^{2}U}{\partial {x}^{2}}$\>Д’Аламбер, 1746 год\\
9.\>Преобразование \>$f(a)=\underset{\infty }{\overset{\infty }{\int }}f\left(x\right){e}^{-2\pi xa}dx$\>Ж. Фурье, 1822 год\\
\>Фурье\\
10.\>Уравнение Навье-\>$p(\frac{\partial v}{\partial t}+\bf v \cdot \nabla \bf v) = - \nabla p + \nabla \cdot \bf T + \bf f $\>К. Навье, Д. Стокс, 1845 год\\
\>Стокса\\
11.\>Уравнения\>$\nabla \cdot \bf E = \frac{p}{\epsilon_0} $\ \ $ \nabla \cdot \bf H = 0$\>Д. К. Максвелл, 1865 год\\
\>Максвелла\>$\nabla  \times E = - \frac{1}{e} \frac{\partial \bf H}{\partial t} $\ \ $\nabla  \times H = \frac{1}{e} \frac{\partial \bf E}{\partial t} $\\
12.\>Второй закон\>$dS\ge 0 $\>Л. Больцман, 1874 год\\
\>термодинамики\\
13.\>Относительность\>$E=mc^2 $\>Эйнштейн, 1905 год\\
14.\>Уравнение \>$ih \frac{\partial}{\partial t} \psi = H \psi $\>Э. Шредингер, 1927 год\\
\>Шредингера\\
15.\>Теория информации\>$H = - \sum p(x) \log p(x) $\>К. Шеннон, 1949 год\\
16.\>Теория Хаоса\>$x_{t+1}=kx_{t}(1-x_{t}) $\>Роберт Мей, 1975 год\\
17.\>Модель Блэка-\>$\frac{1}{2} \sigma^2 S^2 \frac{\partial^2 V}{\partial S^2} + rS \frac{\partial V}{\partial S} + \frac{\partial V}{\partial t} - rV = 0$\>Ф. Блэк, М. Шоулз, 1990 год\\
\>Шоулза
\end{tabbing}

\end{document}
